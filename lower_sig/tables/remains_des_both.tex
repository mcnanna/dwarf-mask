\documentclass[twocolumns,tighten]{aastex61}
\begin{document}
\begin{deluxetable*}{lccccccc}
\tablewidth{0pc}
\tablecolumns{8}
\tablehead{
\colhead{Name} & \colhead{TS} & \colhead{SIG} & \colhead{$\alpha_{2000}$} & \colhead{$\delta_{2000}$} & \colhead{$m - M$} & \colhead{$m - M$} & \colhead{Angular Separation} \\
 \colhead{} & \colhead{(ugali)} & \colhead{(simple)} & \colhead{(deg)} & \colhead{(deg)} & \colhead{(ugali)} & \colhead{(simple)} & \colhead{(deg)}  }
\startdata
J0142.3-3015 & 92.9 & 7.28 & 25.565 & -30.26 & 16.5 & 16.5 & 0.058\\
J0145.0-4335 & 42.6 & 7.75 & 26.262 & -43.59 & 23.0 & 17.0 & 0.0064\\
\enddata
\end{deluxetable*}
\end{document}